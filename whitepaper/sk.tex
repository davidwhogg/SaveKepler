% This file is part of the SaveKepler project.
% Copyright 2013 the authors.

\documentclass[letterpaper,12pt,preprint]{aastex}
\usepackage{enumitem, color}
\definecolor{hypercolor}{RGB}{0,0,127}
\usepackage[%
  citecolor=hypercolor,%
  linkcolor=hypercolor,%
  urlcolor=hypercolor,%
  backref=false,%
  pagebackref=false%
]{hyperref}%

\newcommand{\sectionname}{Section}
\newcommand{\documentname}{\textsl{white paper}}
\newcommand{\foreign}[1]{\textit{#1}}
\newcommand{\vs}{\foreign{vs}}
\newcommand{\etal}{\foreign{et~al.}}
\newcommand{\observatory}[1]{\textsl{#1}}
\newcommand{\Kepler}{\observatory{Kepler}}
\newcommand{\TESS}{\observatory{TESS}}
\newcommand{\SDSS}{\observatory{SDSS}}
\newcommand{\WISE}{\observatory{WISE}}
\newcommand{\project}[1]{\textsl{#1}}
\newcommand{\MAST}{\project{MAST}}
\newcommand{\TheTractor}{\project{The~Tractor}}
\newcommand{\emcee}{\project{emcee}}
\newcounter{hoggitem}
\setcounter{hoggitem}{0}
\newcommand{\hoggitem}{\refstepcounter{hoggitem}\textbf{\textsl{(\thehoggitem)}}}

\newcommand\independent{\protect\mathpalette{\protect\independenT}{\perp}}
\def\independenT#1#2{\mathrel{\rlap{$#1#2$}\mkern2mu{#1#2}}}

\begin{document}\sloppy\sloppypar\thispagestyle{empty}

\title{Maximizing \Kepler\ science return per telemetered pixel: \\
  Detailed models of the focal plane in the two-wheel era}
\noindent
A white paper submitted in response to the \Kepler\ Project Office
\textit{Call for White Papers: Soliciting Community Input for
  Alternate Science Investigations for the Kepler
  Spacecraft}\footnote{\url{http://keplergo.arc.nasa.gov/docs/Kepler-2wheels-call-1.pdf}}
released 2013 August 02.

\begin{description}[style=nextline,itemsep=0ex]
\item[David W. Hogg]
\textit{Center for Cosmology and Particle Physics, New York University}
\item[Ruth Angus]
\textit{Department of Physics, Oxford University}
\item[Tom Barclay]
\textit{NASA Ames Research Center}
\item[Rebekah Dawson]
\textit{Havard--Smithsonian Center for Astrophysics}
\item[Rob Fergus]
\textit{Courant Institute of Mathematical Sciences, New York University}
\item[Dan Foreman-Mackey]
\textit{Center for Cosmology and Particle Physics, New York University}
\item[Michael Hirsch]
\textit{Max-Planck-Institut f\"ur Intelligente Systeme}
\item[Dustin Lang]
\textit{McWilliams Center for Cosmology, Carnegie Mellon University}
\item[Ben Montet]
\textit{Department of Astronomy, California Institute of Technology}
\item[David Schiminovich]
\textit{Department of Astrophysics, Columbia University}
\item[Bernhard Sch\"olkopf]
\textit{Max-Planck-Institut f\"ur Intelligente Systeme}
\end{description}

\clearpage

\section{Executive summary}

\paragraph{primary recommendation:}
\hoggitem~Fundamentally, this \documentname\ advocates \emph{image modeling}%
  ---building a detailed model of pixel sensitivities
  (including possibly intra-pixel sensitivity maps),
  point-spread function,
  and sky, bias, and dark signals,
  all as a function of focal-plane position%
  ---along with a model of the position and brightness of every star in the field.
This level of modeling has not happened with \Kepler\ up to now
  because the data have been made extremely precise with good pointing
  and aperture photometry.
In the two-wheel era, \Kepler\ will not maintain precise pointing.
This is a blessing as well as a curse:
It reduces the precision of naive aperture photometry,
  but it provides data diversity that permits inference
  of the sensitivity map and point-spread function
  that is not perfectly covariant with the morphology of the true scene.
We propose capitalizing on this to
  \textbf{develop a probabilistic generative model of the \Kepler\ pixels.}
We argue that this modeling may permit continuance of photometry at 10-ppm-level precision.
We demonstrate baby steps towards focal-plane models
  along two different directions:
In one, the model is physical;
  its parameters are parameters of the spacecraft, detectors, and optics.
In the other, the model is data-driven but motivated by ideas of causality;
  in this case the parameters control the ways
  disparate data sources can be used to predict one another.
We demonstrate that the expected drift or jitter in positions in the two-weel era
  will \textsl{(a)}~help with constraining these kinds of model,
  and \textsl{(b)}~be obviated (in terms of loss of precision) by such modeling.
These results are relevant to \emph{almost any} scientific goal for the repurposed mission,
  independent of our secondary recommendations.

\paragraph{secondary recommendations:}
\hoggitem~If \Kepler\ continues to observe the current \Kepler\ field,
  we would recommend shortening exposure times to about 5\,min, and increasing cadence,
  to reduce the effect of pointing drift and increase the sensitivity of the models.
Increased cadence will lead to better value of the data for measuring
  transit-timing and transit-duration variations.
If \Kepler\ switches to observing a set of new but specially chosen ``low-torque'' fields,
  it could continue with its current exposure times (of about 30\,min).

\hoggitem~The telemetry burden of any increase in cadence can be offset
  by reducing the number of target stars.
We would recommend (for scientific reasons) aggressively shrinking the target lists around
  bright, non-variable Sun-like and M-type stars.

\hoggitem~It will be imperative to operate the spacecraft
  with one or both of the following modifications to operational software:
Either the pointing will need to be adjusted frequently using
  the two operational wheels and some propellant;
  or else the focal-plane apertures telemetered down at each detector
  read will have to be adapted in real time to follow drifting stars.

\hoggitem~It will also be wise to further capitalize on the ``blessing'' of the drift
  by diversifying the observations in other ways:
Deliberate focus pulls, dithers, and integration-time adjustments,
  either throughout the mission
  or else in specified ``calibration periods''
  will provide much more data support for hard-to-constrain model parameters.

\hoggitem~Whether the two-wheel survey strategy ends up being
  to continue observations in the current \Kepler\ field,
  or to work in the low-torque fields%
  ---we discuss scientific suvey strategies in a companion \documentname\ (by Montet et al.)---%
  it is our view that \Kepler\ can still accomplish
  what many of the present authors consider its key scientific goal,
  which is \textbf{to find Earth-like planets on year-ish orbits around Sun-like stars}.
With a multi-pronged image-modeling effort and
  a bit of good luck
  (with respect to assumptions about spacecraft hardware and data-modeling software),
  it is our view that a two-wheeled \Kepler\ can still
  be used to pursue this deep and important mission.
That said, what's written in this \documentname\ is fundamentally
  agnostic about the scientific program in the two-wheel era.

\clearpage

\section{Philosophy and motivation}

Image modeling will deliver many bits; we will get more science per
downloaded pixel than the naive predictions of degradation.

\Kepler\ data are taken in a mode that is
near-pessimal\footnote{Neologism ``pessimal'' is the antonym of
  ``optimal''.} for the determination or inference of calibration
parameters such as point-spread function, pixel-level or large-scale
sensitivity map, bias, dark, sky, or contributions from overlapping
sources.

\Kepler\ core goals remain outrageously interesting.

We know a lot about the prevalance of planets and the stars in which
we could possibly detect them; we should use this to reduce our data
volume.

TESS is coming; we can help with that.

Asteroseismology and transit timing variations both benefit from
shorter cadence.

If using such fields becomes a requirement of the repurposed mission,
  this would recommend against making year-ish orbit planets a high priority.
  (this isn't necessarily true! If \Kepler\ changes pointings every 20 days, returning to the same pointing once
  per year, it will still be well-positioned to observe transits of certain planets; i.e. planets that transit
  once per year. It will only find 1/~20 of them in each field, but will look at 20 fields per year, so the
  net number of 1AU planets observed will be conserved. Difficulty: 180 day planets will also be observed once/year,
  as will 90, 45, etc. Can be dealt with statistically, through transit durations, and/or with space-based follow up
  (ie JWST) -btm  Okay I agree with this; let's move this to whitepaper 2! -Hogg)

One final point of philosophy:
All of the code used to make the demonstrations in this \documentname
  are available as open-source or at least publicly readable code on the web.%
\footnote{Our toy future-data simulation---and code for physical modeling thereof---is at
  \url{http://astrometry.net/svn/trunk/projects/kepler/};
  it is a sub-project of our general image-modeling framework
  called ``\TheTractor'' (\citealt{hoggtractor}, \citealt{langtractor}), which is at
  \url{http://theTractor.org/}.
  Our data-driven modeling code is at
  \url{https://github.com/dfm/causal-kepler}.
  Code to build physical models of the pixels in the extant \Kepler\ data is at
  \url{https://github.com/dfm/kpsf}.
  Our Python wrapping of the \MAST\ interface to the \Kepler\ data is at
  \url{https://github.com/dfm/kplr},
  and most of our probabilistic inference involves ensemble sampling
  with \emcee\ (\citealt{emcee}), which is at
  \url{http://dan.iel.fm/emcee/}.}
We encourage the \Kepler\ community to fork, build on, and contribute back to these code bases.

\section{Modeling of extant \Kepler\ data}\label{sec:extant}

\subsection{Data-driven conditional models of Kepler data}

The core Kepler science team has developed various methods for removing some
systematic instrumental signals from the extracted light curves.
The set of algorithms called PDC \citep{map-pdc1,map-pdc2} are particularly
successful because they remove instrumental effects (assumed common between
nearby targets on the detector) while retaining astrophysical signals (assumed
unique to the target of interest).
These methods use a data-driven linear model built directly from the aperture
photometry of nearby stars.
Here, we present an extension to this model that takes advantage of the causal
structure of the problem (CITE SOMETHING).
We also argue that this procedure is best performed \emph{at the pixel level}
instead of on the extracted photometric time series.
To start, we describe a model to remove systematic effects from the time
series in a single pixel using the fluxes of pixels from some $K\ge1$ nearby
star(s) and the ``historical'' time series of the target pixel.
The main insight of this method is that the systematic effects are actually
\emph{causally predictable}.
In other words, the (recent) history of the detector sensitivity provides a
very better estimate of the conditions at a given moment than an
instantaneous measurement considered alone.

\paragraph{Implementation}
In this section,
we will discuss the calibration of pixel $i$ of target star $n$ at time $t_k$
using the pixels $j$ on target $m \ne n$ in the sliding window $t_k - \delta t
\le t \le t_k + \delta t$.
As an extension, we can also use the data in pixels $i^\prime$ from the target
$n$ (including pixel $i$) in the windows $t_k - \Delta - \delta t \le t \le
t_k - \Delta$ and $t_k + \Delta \le t \le t_k + \Delta + \delta t $ for
$\Delta \gg $ the timescale of the signal of interest (for example, a
transit).

We start by making the (reasonable) assumption that the observed signal---the
flux measured in pixel $i$ at time $t_k$: $y_i (t_k)$---is a function of only
the following variables:
\begin{itemize}
\item
the ``true'' flux $f_i(t_k)$ arriving at the location of pixel $i$ during the
integration time,

\item
bulk variations caused by the conditions of the spacecraft (for example,
pointing shifts and temperature fluctuations) that affect every pixel in a
systematic---but not identical---way, and

\item
instrumental noise due to the detector.
\end{itemize}
The goal is to find (or model) the true fluxes $f_i(t_k)$ conditioned on the
observations $y_i(t_k)$.
In practice, this is hard because we don't have a physical model accurate at
the required levels.
Instead, we propose an effective model based on the observations described
above.

The efficaciousness of our method relies on the following assumptions:
\begin{itemize}
\item
\emph{Independence}:
we assume that the true fluxes are independent: $f_i(t_k) \independent
f_{i^\prime} (t_{k^\prime})$ for all $i \ne i^\prime$.
We also assume there may be additional observables (for example, temperature)
for which this independence holds.

\item
\emph{Joint confounding}:
we assume that any confounding effect on $y_i(t_k)$ will also affect a number
of other values $y_{i^\prime} (t_{k^\prime})$ (where $i \ne i^\prime$ and
$t_{k^\prime}$ is not necessarily equal to $t_k$).
This means that the overall systematic affecting $y_i(t_j)$ can (in
principle) be predicted from from the other $y_{i^\prime}(t_{k^\prime})$.
\end{itemize}
To motivate our independence assumption, we refer to Reichenbach's principle
\citep{Reichenbach1956}.
It states that whenever we find two observables to be statistically dependent,
then there must be an underlying causal structure responsible for these
dependences.
This structure can be a direct causal link, or an unobserved confounder
affecting both observables.
Vice versa, if there is no causal structure connecting both observables, they
must be statistically independent.

The idea behind our approach is that if we now try to predict a measurement
$y_i (t_k)$ from variables known to be \emph{independent} of $f_i (t_k)$ (the
true flux), then whatever we \emph{can} predict has nothing to do with
$f_i (t_k)$.
Whatever is predictable better than chance---whatever statistical dependence
we find---must, by Reichenbach's principle, be due to joint confounders that
affect both our measurement and the other variables we are using to predict
it.

For simplicity, we assume that the systematic confounders affect the signal
additively but it may be possible to generalize the method.
Under this assumption, the model can be written as
\begin{eqnarray}
y_i(t_k) = f_i (t_k) + \sum_{i^\prime,\,k^\prime} c_{i^\prime\,k^\prime}\,
    y_{i^\prime} (t_{k^\prime}) \quad.
\end{eqnarray}
We propose to estimate the vector of coefficients $\mathbf{c}$ by training a
regression model on the values $y_{i^\prime} (t_{k^\prime})$.
In this model, the true flux $f_i (t_k)$ is considered ``noise''---even though
it the main signal of interest.
This works because (as described above) the signal is independent of the
functions $y_{i^\prime} (t_{k^\prime})$ being used to build the model.

\paragraph{Autoregressive generalization}
The model discussed in the previous section should retain \emph{any true
astrophysical signals} because there is no physical mechanism that would
induce correlations between these signals in different targets.
Suppose that we are only interested in temporally compact astrophysical
signals---exoplanet transits, for example.
In this case, we can substantially improve the predictive power of the model
by using information from the (recent) past of the pixels targeting the star
that we are trying to model.
In this context, a weaker version of the independence assumptions still
applies.
The signal is independent of the data targeting \emph{other stars} and the
observations targeting the target of interest for times $t_{k^\prime}$ where
$|t_k - t_{k^\prime}| > \Delta$ where $\Delta$ is longer than the maximum
duration of a transit.
This independence assumption also has a physical interpretation: in addition
to all the above independences, the star's light emission is independent of
where the planet is located around the star.
Therefore, the light arriving from the star system contains no information
about the transit signal except while the planet is in front of (or behind)
the star.
This can be viewed as an assumption of independence of the stellar limb
darkening profile and the geometric mechanism by which the
star-planet system ``converts'' the star brightness into an observable
brightness of the joint system.
This kind of assumption has recently studied in the field of causal inference
by \citet{JanSch10}.

\section{Toy simulations of future data}\label{sec:future}

HOGG IS STILL MERGING THIS SECTION IN

Motivated by the ``371-roll'' scenario (HOGG CITATION),
  we produced a series of toy (that is, unrealistic in various ways)
  simulated 30-minute coadd images with $\sim$ half-pixel drift and 1~arcsec jitter.
Drift and jitter were simulated by moving the boresight of the simulated telescope every 6 seconds during the integration.  We examine an aperture around a target, where the aperture is large enough to contain the full drift path of the target over the 24-hour observing window.  The stars in the simulation are real Kepler input catalog sources.  The point-spread function is a very much simplified double-Gaussian.  We introduced variations in the flat-field, drawn pixelwise from a Gaussian with 1% standard deviation.  We produced a series of 48 coadds, corresponding to one day's observations between rolls.

In order to simplify the modeling task, in addition to the science exposure we rendered a ``perfect'', bright, isolated, noise-free star with an unperturbed flat-field.  The star is still drifted and jittered in each coadd.  We use this ideal star to learn the PSF model in each coadd.  For a PSF model we use a very simple mixture of 4 Gaussians.  See figure 1.

Fig 1 (first three plots): (a) ``perfect'' PSF simulated image, shown on a log scale with a dynamic range of $10^6$. (b) Model PSFs.  These are 4-component Gaussian mixtures (23 parameters total).  (c) PSF model residuals.  The stretch is $\pm 10^{-4}$.

After learning a PSF model for each image, we infer the brightness of each source in the field.  We assume we have the true source positions from the Kepler input catalog (the same positions used to produce the simulations), and the predicted drift path.

Fig 2 (next two plots): Simulated and modelled images.  The model images are created by fitting a PSF model for each image, and then inferring the brightness of each source.  The three images shown are the first, middle, and last images in the 48-image sequence.

After fitting for the brightness of each source, we estimate the flat field.  We do this in a very simple way, by taking the geometric mean of the ratio of the model to simulated images.  The true and recovered flat-fields are shown in figure 3.

Fig 3 (next two plots): Flat-field recovery.  The true flat-field in the image has Gaussian variation with 1% standard deviation.  We estimate the flat-field based on the mean pixelwise ratio between the model image and (simulated) data.

My plan, on the off chance I get some more time to work on this today, is to make a true flat-field with twice the resolution of the data -- subpixel sensitivity variations, with 4 regions per pixel -- and see if I can infer it. Simple enough, conceptually: render the model at twice the resolution, multiply by flat-field, and sum 2x2 to produce the original-resolution image.  Optimize the 4 x number-of-pixels subpixel flat-field vector to best match the observed image.

\section{Target selection}\label{sec:target}

\section{Spacecraft pointing}\label{sec:pointing}

\section{Aperture selection for telemetry}\label{sec:telemetry}

\section{Calibration strategy}\label{sec:calibration}

Will self-cal be enough?

Should we pull focus and dither and exercise exposure time options?

\section{Possible showstoppers}\label{sec:stop}

In principle any physical effect can be covered by the modeling
methods we propose.

Heterogeneity or time dependence in intrapixel sensitivity variations
could be very difficult to model.

Propose multipole expansion to begin to address this.

\section{Acknowledgments}

DFM would like to thank Sameer Agarwal (Google) and Keir Mierle (Locu) for
releasing the Ceres non-linear least squares solver
(\url{https://code.google.com/p/ceres-solver/}) and for making suggestions
that enabled this work.

% ---
% despite what astronomers say, let's put titles in the references
% ---
\begin{thebibliography}{}\raggedright%

\bibitem[Foreman-Mackey \etal(2013)]{emcee}
Foreman-Mackey,~D., Hogg,~D.~W., Lang,~D., \& Goodman,~J., 2013,
\emcee:\ The MCMC Hammer,
\pasp, 125, 306--312.

\bibitem[Hogg \& Lang(2013)]{hoggtractor}
Hogg,~D.~W. \& Lang,~D., 2013,
Replacing standard galaxy profiles with mixtures of Gaussians,
\pasp, 125, 719--730

\bibitem[Lang \& Hogg(2014)]{langtractor}
Lang,~D. \& Hogg,~D.~W., 2014,
\TheTractor:\ A framework for image modeling, with applications to \SDSS, \WISE, and \Kepler,
in preparation

\bibitem[Janzing \& Sch{\"o}lkopf(2010)]{JanSch10}
Janzing,~D.\ \& Sch{\"o}lkopf,~B., 2010,
% DFM NEED TITLE
IEEE Transactions on Information
Theory, 56, 5168

\bibitem[Reichenbach(1956)]{Reichenbach1956}
Reichenbach,~H., \emph{The Direction of Time}, University of California
Press, 1956

\bibitem[Smith \etal(2012)]{map-pdc2}
Smith, J.~C., Stumpe, M.~C., Van Cleve, J.~E., \etal\ 2012, \pasp, 124, 1000

\bibitem[Stumpe \etal(2012)]{map-pdc1}
Stumpe, M.~C., Smith, J.~C., Van Cleve, J.~E., \etal\ 2012, \pasp, 124, 985

\end{thebibliography}

\end{document}
