%\documentclass{emulateapj}
\documentclass[12pt, preprint]{aastex}
%\usepackage{natbib,graphicx,amsmath,amsthm,ulem,color, lscape, caption,subfigure}
\usepackage{natbib,graphicx,amsmath,amsthm,ulem, subfigure, enumitem}



\newcommand{\sectionname}{Section}
\newcommand{\documentname}{\textsl{white paper}}
\newcommand{\foreign}[1]{\textit{#1}}
\newcommand{\vs}{\foreign{vs}}
\newcommand{\observatory}[1]{\textsl{#1}}
\newcommand{\kepler}{\observatory{Kepler}}
\newcommand{\Kepler}{\kepler}
\newcounter{hoggitem}
\setcounter{hoggitem}{0}
\newcommand{\hoggitem}{\refstepcounter{hoggitem}\textbf{\textsl{(\thehoggitem)}}}

\newcommand\independent{\protect\mathpalette{\protect\independenT}{\perp}}
\def\independenT#1#2{\mathrel{\rlap{$#1#2$}\mkern2mu{#1#2}}}


\shorttitle{Save Kepler 2}
\shortauthors{Montet et al.}

\bibliographystyle{apj}
\begin{document}\sloppy\sloppypar\thispagestyle{empty}

\title{Maximizing \Kepler\ science return per telemetered pixel: \\
  Searching the habitable zones of the brightest stars}
\noindent
A white paper submitted in response to the \Kepler\ Project Office
\textit{Call for White Papers: Soliciting Community Input for
  Alternate Science Investigations for the Kepler
  Spacecraft}\footnote{\url{http://keplergo.arc.nasa.gov/docs/Kepler-2wheels-call-1.pdf}}
released 2013 August 02.


\begin{description}[style=nextline,itemsep=0ex]
\item[Benjamin T. Montet]
\textit{Department of Astronomy, California Institute of Technology}
\item[Tom Barclay]
\textit{NASA Ames Research Center}
\item[Rebekah Dawson]
\textit{Havard--Smithsonian Center for Astrophysics}
\item[Rob Fergus]
\textit{Courant Institute of Mathematical Sciences, New York University}
\item[Dan Foreman-Mackey]
\textit{Center for Cosmology and Particle Physics, New York University}
\item[Michael Hirsch]
\textit{Max-Planck-Institut f\"ur Intelligente Systeme}
\item[David W. Hogg]
\textit{Center for Cosmology and Particle Physics, New York University}
\item[Dustin Lang]
\textit{McWilliams Center for Cosmology, Carnegie Mellon University}
\item[David Schiminovich]
\textit{Department of Astrophysics, Columbia University}
\item[Bernhard Sch\"olkopf]
\textit{Max-Planck-Institut f\"ur Intelligente Systeme}
\end{description}


\section{Executive Summary}

\paragraph{primary recommendation:}
In Paper I (Hogg et al.) we propose image modeling techniques to 
  maintain 10-ppm-level precision photometry even with only two working 
  reaction wheels. 
While these results are relevant to many scientific goals
  for the repurposed mission, we advocate for a change of strategy; namely, 
  \textbf{observing bright stars continuously in ``low-torque'' fields across the ecliptic plane.}
There are considerable benefits of such a strategy. 
\begin{itemize}
\item
The first scienfitic goal\footnote{\url{http://kepler.nasa.gov/science/about/scientificgoals/}} of \Kepler\ is to \textbf{determine the frequency 
 of terrestrial and larger planets in or near the habitable zone of a wide 
 variety of spectral types of stars}. 
Not only does this recommendation not detract from this strategy, it may 
 provide the best chance to answer this question moving forward.
\item
By limiting ourselves to bright stars, the \kepler\ long cadence integrations of 29.4 minutes can be shortened, allowing for more sensitive observations of transit timing and transit duration variations.
These stars will be later observed by TESS, but only for 30 days; there will not be enough transits observed to detect variations due to planet-planet interactions.
Therefore, by observing bright stars that can be followed up by TESS, this will enable dynamical studies that could not be undertaken with either telescope alone.
\item
Similarly, shorter integrations may allow for asteroseismological studies of 
 more stars. 
Many asteroseismological targets have been subgiant or giant stars, for which 
 the period of oscillations are long enough so that they can be observed in 
 long cadence data (Chaplin Ann Rev. 2013)
By decreasing the integration time, solar-type oscillations can be observed 
 on less massive stars.
On these stars, asteroseismological signals are smaller, but by focusing on 
 bright stars we expect the signals to still be observable.
\end{itemize}
This white paper is organized as follows. In \textsection\ref{Targets}, we outline our target selection and observing strategies. In \textsection\ref{Haul}, we project the expected number of planets detected by this strategy. In \textsection{HZ}, we explain how this strategy will enable completion of the primary scientific goal of the \Kepler\ mission. \textsection\ref{Dynamics} and \textsection\ref{AS} outline the dynamical and asteroseismological ``bonus science'' that can be accomplished, the former especially in a synergistic manner with TESS. 

\section{Target Selection}
\label{Targets}
Simulations suggest pointing errors from a two wheel \Kepler\ will be 
 minimized by pointing in the ecliptic plane, where the torque exerted on 
 the telescope by solar pressure is approximately zero. 
Such areas are expected to be stable for approximately six months, so that the 
 expected drift is negligible over a 30 minute or shorter integration.
We propose to observe fields in the ecliptic plane for this reason. 
To avoid the Earth crossing the field of view, it is recommended the telescope
 point only in the direction opposite motion, meaning each field is stable 
 \emph{and observable} for only three months.
\textbf{Fortunately, the ecliptic plane is ideal for the \Kepler\ telescope}.
Recall the \Kepler\ field is situated 8-18 degrees out of the galactic plane.
This location provides a compromise between having a sufficent quantity of 
 $K_p < 16$ stars to observe and a lack of crowding in the telescope's 4 
 arcsecond pixels. 
The ecliptic plane and galactic plane cross twice, at RA of approximately
 7 and 19. 
It is therefore possible to choose four fields, each separated by
 approximately 6 hours in RA, that are both on the ecliptic plane and
 approximately the same distance from the galactic plane as the \Kepler\
 field. 
These are located near RA of 3, 9, 15, and 21. 

In this paper, we do not attempt to choose specific fields, but instead
 simply show that there are sufficent observable stars such that our proposed
 observations are feasible.
In part, this is because as Hogg et al. discuss, our image modeling techniques
 may be more successful when the drift is larger. 
With degraded pointing and additional drift, the diversity of stars that touch
 different combinations of pixels is increased. 
More simulations are required to determine the optimal positioning of the 
 telescope with respect to the ecliptic so that the drift is large enough 
 to maximize our abilities to model pixel sensitivities but small enough 
 that the stars stay within their apertures.
Fortunately, it is quite simple to select four fields that are both near the 
 ecliptic and well separated such that each can be observed for approximately
 three months.

We recommend shorter integration times than the long cadence observations, 
 as we discuss more fully below. 
As a result, fewer targets are available for observations than have been 
 previously observed. 
We aim for approximately \emph{thirty thousand targets per field}, with 
 preference given to brighter stars. 
Here, ``brighter'' is intentionally left as an ambiguous term. If our goal is to detect Earth-sized planets, then M dwarfs are allowed
 to be fainter than G dwarfs because of their deeper transit signal. 

If our image modeling techniques are successful, then a field similar to the 
 original \Kepler\ field with respect to the galactic plane will contain 
 a sufficent number of viable solar-type and smaller targets. 
To verify each field will contain enough stars, we simulate the population 
 of the nearby galaxy with TRILEGAL (Girardi et al. 2002). 

We first consider the current \Kepler\ field. 
TRILEGAL is limited to fields of 10 square degrees. 
To combat this, we simulate three regions of 8 square degrees across the 
 \Kepler\ field orthogonal to the galactic plane. 
We then estimate the extinction in the galaxy by calibrating the extinction 
 value at infinity using the values found in the extinction map created by 
 Schlegel, Finkbeiner, and Davis (1998). 
We return the \Kepler\ magnitude of each star in our field.

Fig. \ref{Kep1} shows the distribution of stars as a function of effective 
 temperature and apparent magnitude. 
Ideally, stars in the lower left region of the figure would be chosen. 
The number of simulated stars found is comparable with the true number of 
 observable stars in the \Kepler\ field.

There are sufficent hypothetical fields near both the ecliptic and galactic 
 planes to develop four similar fields from which 30,000 stars could each 
 be selected. 
It is of considerable importance, however, that \textbf{if new fields are 
 selected, they should be located a similar distance from the galactic plane}. 
We provide evidence in support of this claim in the form of Fig. \ref{Kep2}. 
This figure was developed in an identical manner to Fig. \ref{Kep1}, but 
 the numbers correspond to the expected yield in a region of the sky far away
 from the galactic plane (0:00:00, +0:00:00, blue) and very close to the center
 of the galactic plane, pointing away from the galactic center (6:00:00, 
 +23:30:00, red). 
Out of the galactic plane, solar-type stars are an order of magnitude less 
 common; in it, these stars are a factor of two more common, meaning crowding
 (and background eclipsing binary false positives) would be even more a concern
 than in the original misison\footnote{This is not a problem for a future M 
 dwarf mission: because the stars are intrinsically faint, most M dwarfs 
 brighter than 16th magnitude are within 100 parsec and distributed 
 approximately isotropically across the sky.}.
Therefore, if new fields are selected we \emph{strongly recommend} selecting 
 four fields near the ecliptic plane and approximately 15 degrees from the 
 galactic plane. 
 
 



\section{Expected Planet Detections}
\label{Haul}
To determine the number of planets we would expect to find with this new 
 observing strategy, we consider the best source of information on transiting 
 exoplanets, the \Kepler\ list of planet candidates found on the NASA 
 Exoplanet Archive\footnote{\url{http://exoplanetarchive.ipac.caltech.edu}}.
The number of detections depends on the level of success of our image modeling
 efforts, which has not yet been precisely quantified. 
A \textit{reasonable, conservative} estimate is that we can expect to detect 
 planets larger than 3 Earth radii around Sun-like stars. 
Planet transit signals induced by these planets would be, of course, nine times
 deeper than the transit of an Earth analogue, and is detectable by eye in the 
 current \Kepler\ data. 

By our mission design strategy, we would stare at each of four fields for one 
 quarter of a year, or 93 days. 
To observe three transits of a planet, we would then be limited to planets with 
 periods of 31 days or smaller. 
We search the Exoplanet Archive for objects that are listed as ``candidates'' 
 with periods less than 31 days and transit depths larger than 756 parts per 
 million, 9 times that of an Earth analogue. 
In the Q1-Q6 catalogue, there are 620 matches to these criteria (in the full 
 catalogue, this number increases by less than 10 percent). 
 
Since we propose to observe 120,000 stars, not the 190,000 observed in the 
 primary mission, we scale this number and find we would expect to detect 
 \textbf{392} candidate planets (after a similar amount of vetting for false 
 positives to the current mission! We would expect to discard 165
 false positives) with periods shorter than 31 days and 
 radii larger than thrice Earth's. 
If our image modeling is successful, these are conservative predictions.
If we are able to detect objects twice the size of Earth, the number of 
 expected candidates jumps to \textbf{749}.
Moreover, this thought experiment assumes a one year new mission. 
If the  remaining two reaction wheels encounter no problems, we would expect 
 to be able to push our detections to longer periods. 
Moreover, we would be biased toward planets with approximately one year periods
 due to our unique observing cadence. 
The proposed cadence is both a blessing and a curse for finding planets in the 
 so-called habitable zone, as we explain in the next section.
 




\section{The Frequency of ``Habitable-Zone'' Planets}
\label{HZ}
Goal 1 of the \Kepler\ mission is the determine the frequency of terrestrial 
 and larger planets in or near the habitable zone of a wide variety of 
 spectral types of stars. 
For M dwarfs the habitable zone varies considerably by spectral type,
 but is always located at periods inside of 30 days (\emph{citation} goes 
 here).
Thus, in a one-year mission we would expect to determine the frequency of 
 M dwarf habitable systems. This number is presently poorly constrained due 
 to the small number of M dwarfs observed in the primary mission; we hope
 this is given prime consideration in the future plans for the telescope.

The true frequency of habitable zone planets is less constrained for solar-type 
 stars.
If our image modeling techniques are successful, such a strategy as the 
 one proposed here will enable an estimate of this value in a multi-year 
 extended mission. 
Because we return to the same field every year, we will be biased towards 
 planets that transit once per year. 
Planets that transit exactly twice (or three, or $N$) times per year will 
 also appear in our sample as habitable zone ``imposters.'' 
Such imposters can be accounted for in a statistical sense by an analysis of 
 transit durations (which increase as a function of period), and for 
 individual highly interesting systems by future space-based follow up. 
Therefore, we do not expect such ``imposters'' to be a significant hinderance.

For each field, because of our cadence, we expect to miss a significant 
 fraction (approximately $75\%$) of the transiting habitable zone planets. 
This is an unavoidable effect from this observing strategy, but should 
 not affect our results. 
Because \Kepler\ would always be staring at one of the four proposed fields, 
 it will always be on the lookout for \emph{some} transiting, potentially 
 habitable systems; this missed systems can be accounted for statistically.


\section{Dynamics with \Kepler\ and TESS}
\label{Dynamics}
\subsection{The Flatness of Transiting Exoplanetary Systems}
Despite the tremendous success of \Kepler\ at finding planets, questions remain
 as to the flatness of the exoplanetary systems uncovered. 
While the vast majority of transiting systems must have inclinations of a 
 few degrees or fewer, it is unclear how flat is ``flat.'' 
Moreover, in special cases we may expect inclined companions to transiting 
 planets. 
For example, hot Jupiters may be formed by early dynamical interactions with a 
 mutually inclined perturber, initially exchanging inclination for eccentricity
 and then circularizing through tidal effects. 
If this scenario is accurate, then we would expect inclined companions to 
 slowly perturb transiting planets. 
As the inclination of a transiting planet changes, so does the chord the planet
 cuts across the face of its star. This also necessarily changes the 
 \emph{duration} of the planet transit, especially for large impact parameters
 when a small change in inclination significantly affects the length of the 
 transit chord.

Due to its four year mission lifetime, \Kepler\ is not optimal for observing 
 these secular effects, which occur with timescales 
\begin{equation}
\tau = \frac{M_\star}{M_p}P_{tr}
\end{equation}
with $M_\star$ the mass of the host star, $M_p$ the perturber's mass, and
 $P_{tr}$ the period of the transiting planet. 
For a planet in a 10-day orbit perturbed by a $2 M_J$ object, this cycle is 
 approximately 14 years. 
 However, a repurposed \Kepler\ working in concert with
 TESS or a future TESS-like mission would be the ideal instrument for this 
 study. 
\Kepler\ will be able to measure the transit durations of large, warm 
 transiting planets in a few fields in 2014. 
Moreover, with an increased cadence the precision of observations will be 
 enhanced over the work currently accomplished in the \Kepler\ field. 
Observing transits each year will enable a search for transit duration 
 variations (TDVs). 
When these fields are revisited in 2018-2019 with TESS, we will immediately have
 at least a five year baseline to compare transit durations against. 
\textbf{This is a science objective that could not be carried out by TESS or 
 \Kepler\ alone}. 
A search for TDVs will enable key outstanding questions about the architecture
 of exoplanetary systems to be answered;  there are not any missions current or
 planned that will be able to study the flatness of exoplanetary systems as 
 well as the combination of a repurposed \Kepler\ and TESS
\emph{Perhaps add in a line about apsidal precession? Exomoons?}
 

\subsection{Masses of Transiting Planets}
An unexpected success of the \Kepler\ mission has been the discovery of many 
 systems with tightly-packed inner planets (hereafter STIPs, e.g. Boley 
 and Ford 2013). 
Ten percent or more of stars appear to have planets a few Earth radii in size
 with periods smaller than 20 days. 
These systems appear to form around stars of a variety of spectral types, from 
 Solar-type stars (Kepler-11, Lissauer et al. 2011) to mid M dwarfs. 
 (Kepler-42, Muirhead et al. 2012).
While their existence is unquestioned, their formation is uncertain. 
Boley \& Ford (2013) propose \textit{in-situ} formation via aerodynamic drift, 
 while Cossou et al. (2013) and Swift et al. (2013) propose migration of 
 planetary embryos. 
To better understand the formation and evolution of these systems, it 
 is imperative to understand their mass (and thus density) distributions. 
The most effective method to determine masses of small transiting planets 
 to date has been the characterization of transit timing variations (TTVs) 
 (e.g. Fabrycky et al. 2012), the effects of gravitational perturbations
 between planets near each conjunction. 

Far and away the most common type of TTV observed with 
 \Kepler\ is variations that occur on the timescale of planetary conjunctions
 for near-resonant systems (the ones described above). 
A mission that observes systems for approximately one month per year is 
 suboptimal for detailed characterization of this type of TTV for specific 
 systems. 
However, all hope is not lost.
In the commonly observed case where both near-resonant planets transit, the 
 period of the TTV signal is known and equal to the period of conjunctions,
 leaving the only free parameters the amplitude of the TTV signal and its phase.
As the typical period of these conjunctions is 1-2 years, one quarter of
 observations is not enough to determine the masses of the transiting planets. 
However, return observations of these systems over multiple years with \Kepler\
 and eventually with TESS can enable a full cycle to be observed, allowing for 
 masses to be estimated. Even when a full cycle can not be observed, strong 
 upper limits can be placed on the masses of the planets from TTV nondetections.

\Kepler\ was, fundamentally, a statistical mission. 
A study such as this will  help us better understand the statistics of TTV 
 systems. 
While we will likely not be able to characterize systems as well as can be
 presently accomplished  (e.g. KOI-142, Nesvorny et al. 2013), statistical
 analyses that lack sufficent numbers of TTV systems will be able to be 
 undertaken. 
If the updated \Kepler\ mission collects four pointings each of 30,000 stars, 
 then of the 120,000 stars observed, 12,000 would be expected to host STIPs
 and approximately 600 of these systems would be expected to trasit.
Adding numbers such as these to the current sample will allow a tremendous 
 increase in our understanding of the properties of these systems and provide
 insight into their formation.


 

\section{Other Benefits}
\label{AS}
In some ways, the proposed mission is similar to a mini-TESS. 
There are considerable benefits to such a strategy. 
Many have been outlined above; the potential to open up TESS to searching for 
 TDVs and TTVs over 400 square degrees of the sky cannot be overstated. 
TESS will also have the ability to search for longer-period planets in these 
 fields, when its data is combined with \Kepler\ data.
It is worth mentioning that this proposed survey will only cover one percent 
 of the sky, and will thus not impinge on TESS' discovery space in any 
 significant way. 
It is of our opinion that the benefits to observing these fields in the 
 first half of this decade as a precursor to TESS only serves to enhance 
 the future mission, as opposed to detracting from it.

Observing brighter stars at a higher cadence will allow asteroseismological 
 studies of more stars than were observable with the completed
  \Kepler\ mission. 
With decreasing stellar density, both the magnitude of the pulsations and 
 their periods increase, making giants more appropriate asteroseismological  
 targets. With brighter targets at a higher cadence, if carefully planned, 
 we can open up these studies to stars nearer the main sequence. \textbf{Tom, 
 this discussion is very superficial, perhaps because that is the best word to 
 describe my knowledge of asteroseismology. Does Dan Huber want to add 
 something here?}

An additional benefit to focusing on brighter stars is the increased ability 
 for radial velocity (RV) followup. 
If our proposed obsering strategy is undertaken, there will
 be four new \Kepler\ fields, two near a declination of +20 degrees and two 
 near -20 degrees. 
These are ideally placed for follow up radial velocity work
 by existing telescopes in Hawaii and Chile, as well as future 30-meter class
 telescopes. 
The faintest stars in the \Kepler\ field are too faint to be observed 
 effeciently on existing telescopes, and with observing time as competitive 
 as it will likely be on the proposed giant telescopes, this is unlikely to 
 change.
Therefore, a focus on brighter stars (perhaps $K_p < 15$ for solar-type stars) 
 will enable more effecient and successful RV follow up.
 
Finally, and perhaps most significantly, such a mission will allow potentially 
 interesting objects to be found before the launch of the James Webb Space 
 Telescope (JWST). 
Scheduled to launch in \sout{2011} \sout{2015} 2018, JWST will be placed at 
 the inaccessable L2 Lagrange point, making it necessarily a fixed-length 
 mission. 
Moreover, by this time the \textit{Spitzer} telescope will have drifted too 
 far from Earth to still be scientifically useful.
TESS is scheduled to launch at approximately the same time as JWST, but 
 space telescopes do occassionally run into delays for various reasons\footnote{see also: JWST}.
If TESS were to be delayed, and JWST ran for only its nominal five year mission, it is conceivable that many of TESS' most interesting discoveries will occur 
 at a time when there are no available infrared space facilities available for 
 follow up work.
Certainly, even if both telescopes proceed according to plan, time will be 
 limited between publication of TESS' results and the end of the JWST mission.
This proposal, which intends to find interesting transiting systems across 
 the sky, will accomplish this feat well before the launch of JWST, allowing 
 considerable time to plan observations with the future observatory before 
 launch and assuring that JWST will have an abundance of planets to 
 characterize.

\section{Summary and Conclusions}
Our strategy outlined here is one possible observing strategy. 
If our image  modeling techniques are successful, we feel such a strategy 
 provides a unique opportunity for the contributions of \Kepler\ to continue 
 unhindered for the next decade, through a combination of its observations and 
 those planned in the future by TESS. 

Blah blah blah \textit{Anyone else want to contribute a writeup?}

\section{Acknowledments}
Blatant stealing of text from Hogg's paper goes here.

\end{document}
