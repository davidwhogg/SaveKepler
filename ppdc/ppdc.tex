% This file is part of the SaveKepler project.
% Copyright 2013 the authors.

% to-do
% - write abstract
% - write introduction
% - write method
% - perform and write up experiments
% - write discussion
% - make sure all the references have been cited

\documentclass[letterpaper,12pt,preprint]{aastex}
\usepackage{enumitem, color}
\definecolor{hypercolor}{RGB}{0,0,127}
\usepackage[%
  citecolor=hypercolor,%
  linkcolor=hypercolor,%
  urlcolor=hypercolor,%
  backref=false,%
  pagebackref=false%
]{hyperref}%

\newcommand{\sectionname}{Section}
\newcommand{\documentname}{\textsl{white paper}}
\newcommand{\foreign}[1]{\textit{#1}}
\newcommand{\vs}{\foreign{vs}}
\newcommand{\etal}{\foreign{et~al.}}
\newcommand{\observatory}[1]{\textsl{#1}}
\newcommand{\Kepler}{\observatory{Kepler}}
\newcommand{\TESS}{\observatory{TESS}}
\newcommand{\SDSS}{\observatory{SDSS}}
\newcommand{\WISE}{\observatory{WISE}}
\newcommand{\project}[1]{\textsl{#1}}
\newcommand{\MAST}{\project{MAST}}
\newcommand{\kplr}{\project{kplr}}
\newcommand{\TheTractor}{\project{The~Tractor}}
\newcommand{\emcee}{\project{emcee}}
\newcounter{inlineitem}
\setcounter{inlineitem}{0}
\newcommand{\inlineitem}{\refstepcounter{inlineitem}{\textsl{(\theinlineitem)}}}
\newcounter{address}
\setlength{\parskip}{0ex}

\newcommand\independent{\protect\mathpalette{\protect\independenT}{\perp}}
\def\independenT#1#2{\mathrel{\rlap{$#1#2$}\mkern2mu{#1#2}}}

\begin{document}\sloppy\sloppypar\thispagestyle{empty}

\title{Improving \Kepler\ photometry: \\
  A data-driven model of the pixels for pre-search data conditioning}

\author{%
  Dan~Foreman-Mackey\altaffilmark{\ref{CCPP},\ref{email}},
  David~W.~Hogg\altaffilmark{\ref{CCPP},\ref{MPIA}},
  Rob~Fergus\altaffilmark{\ref{Courant}},
  Stefan~Harmeling\altaffilmark{\ref{MPIIS}},
  Dustin~Lang\altaffilmark{\ref{CMU}},
  Bernhard~Sch\"olkopf\altaffilmark{\ref{MPIIS}},
  others%
}

\setcounter{address}{1}
\altaffiltext{\theaddress}{\stepcounter{address}\label{CCPP}%
  Center for Cosmology and Particle Physics, Department of Physics, New York University}
\altaffiltext{\theaddress}{\stepcounter{address}\label{email}%
  To whom correspondence should be addressed; \texttt{<danfm [at] nyu.edu>}.}
\altaffiltext{\theaddress}{\stepcounter{address}\label{MPIA}%
  Max-Planck-Institut f\"ur Astronomie, Heidelberg, Germany}
%\altaffiltext{\theaddress}{\stepcounter{address}\label{Oxford}%
%  Department of Physics, Oxford University}
%\altaffiltext{\theaddress}{\stepcounter{address}\label{Ames}%
%  NASA Ames Research Center}
%\altaffiltext{\theaddress}{\stepcounter{address}\label{CfA}%
%  Harvard--Smithsonian Center for Astrophysics}
\altaffiltext{\theaddress}{\stepcounter{address}\label{Courant}%
  Courant Institute of Mathematical Sciences, New York University}
\altaffiltext{\theaddress}{\stepcounter{address}\label{MPIIS}%
  Max-Planck-Institut f\"ur Intelligente Systeme, T\"ubingen}
%\altaffiltext{\theaddress}{\stepcounter{address}\label{UCL}%
%  Department of Physics and Astronomy, University College London}
\altaffiltext{\theaddress}{\stepcounter{address}\label{CMU}%
  McWilliams Center for Cosmology, Carnegie Mellon University}
%\altaffiltext{\theaddress}{\stepcounter{address}\label{Caltech}%
%  Department of Astronomy, California Institute of Technology}
%\altaffiltext{\theaddress}{\stepcounter{address}\label{Columbia}%
%  Department of Astronomy, Columbia University}

\begin{abstract}
There are many tiny features in \Kepler\ light-curves
  that are thought to be caused by small changes
  in spacecraft pointing, orientation, and temperature.
These features are small,
  but much larger than the transit signals of greatest interest;
  they also mask or confuse intrinsic stellar variability signals.
Because they are spacecraft-induced, the features can be seen repeated
  across many pixels across the focal plane,
  albeit with different amplitudes and signs.
Inspired by ideas from causal inference,
  we learn a model
  in which the brightness recorded at time $t$ from any target pixel $j$
  is predicted using a weighted linear combination of many other pixel recordings
  spatially separated from $j$ (so they don't overlap the same star)
  and in a finite time window around $t$.
We use a train-and-test framework to avoid over-fitting.
The model produces photometric light-curves ``cleaned''
  of spacecraft-induced variability,
  ideal for studying stellar variability or searching for exoplanet transits.
We learn also a second model that uses,
  in addition to spatially separated pixel recordings,
  nearby and cospatial pixel recordings,
  but separated by time lags longer than expected transit signals.
These latter models remove stellar variability signals
  along with spacecraft-induced signals;
  they are even better for pure exoplanet searching.
We demonstrate the methods on a few different \Kepler\ targets
  and release open-source code.
\end{abstract}

\section{Introduction}

In 2013~May,
  the \Kepler\ Satellite (CITE)%
  ---having discovered thousands of confirmed and likely exoplanets around other stars---%
  lost the use of the second of four of its reaction wheels,
  rendering it substantially impaired.
In 2013~August,
  the \Kepler\ Project Office asked for help and suggestions from the community; and
  in 2013~September many groups submitted white papers.
One of the suggestions we made (\citealt{whitepaper}) to the \Kepler\ Project Office
  was to improve the precision of the mission photometry
  by building predictive models of the pixel values read out by the device.
We were thinking about operations in the ``two-wheel era'',
  in which (in the absence of good image modeling)
  pointing inaccuracy leads to photometric inaccuracy;
  we argued that photometric accuracy could be restored with a good model
  of the (drifting) spacecraft attitude, the focal plane, and the device pixels.
In the process of testing these ideas,
  we discovered that we can perhaps improve the \emph{extant} \Kepler\ data
  as much as we might improve any future data.
Here we present one of these discoveries.

...The PDC is useful but not perfect...give an example or two...

...What does the PDC lack or assume?..

...Here we are going to do what?..

\section{Generalities}

...What's the big idea, and how does it flow from ideas about causal inference?..

...Set up the general frameworks (especially notation) for the experiments section...

\section{Experiments}

\section{Discussion}

\acknowledgements
It is a pleasure to thank...full SaveKepler team, exoSAMSI team, grants.

\begin{thebibliography}{}\raggedright%

\bibitem[Bryson \etal(2010)]{bryson2010}
Bryson,~S.~T., Tenenbaum,~P., Jenkins,~J.~M., \etal, 2010,
The \Kepler\ Pixel Response Function,
\apjl, 713 L97

\bibitem[Esteves \etal(2013)]{esteves2013}
Esteves,~L.~J., De Mooij,~E.~J.~W., Jayawardhana,~R., \etal, 2013,
Optical Phase Curves of \Kepler\ Exoplanets,
\apjl, 772 51E

\bibitem[Foreman-Mackey \etal(2013)]{emcee}
Foreman-Mackey,~D., Hogg,~D.~W., Lang,~D., \& Goodman,~J., 2013,
\emcee:\ The MCMC Hammer,
\pasp, 125, 306

\bibitem[Gautier \etal(2012)]{kepler20}
Gautier,~T.~N.,~III, Charbonneau,~D., Rowe,~J.~F., \etal, 2012,
\Kepler-20: A Sun-like Star with Three Sub-Neptune Exoplanets and Two
Earth-size Candidates
\apj, 749, 15

\bibitem[Gilliland \etal(2011)]{gilliland2011}
Gilliland,~R.~L., Chaplin,~W.~J., Dunham,~E.~W., \etal, 2011,
\Kepler\ mission stellar and instrument noise properties,
\apj, 197, 1

\bibitem[Hogg \etal(2013)]{whitepaper}
Hogg,~D.~W. \etal, 2013,
Maximizing \Kepler\ science return per telemetered pixel:\ Detailed models of the focal plane in the two-wheel era,
arXiv, 1309.0653

\bibitem[Janzing \& Sch{\"o}lkopf(2010)]{JanSch10}
Janzing,~D.\ \& Sch{\"o}lkopf,~B., 2010,
Causal inference using the algorithmic Markov condition,
IEEE Transactions on Information Theory, 56, 5168

\bibitem[Krizhevsky \etal(1998)]{Kriz12}
Krizhevsky,~A., Sutskever,~I., \& Hinton, G.E., 2012,
ImageNet Classification with Deep Convolutional Neural Networks,
Neural Information Processing Systems (NIPS 2012)

\bibitem[LeCun \etal(1998)]{LeCun1998}
LeCun,~Y., Bottou,~L., Bengio,~Y., \& Haffner, P., 1998,
Gradient-Based Learning Applied to Document Recognition,
Proceedings of the IEEE, 86, 2278

\bibitem[Reichenbach(1956)]{Reichenbach1956}
Reichenbach,~H., 1956
\textit{The Direction of Time},
University of California Press

\bibitem[Smith \etal(2012)]{map-pdc2}
Smith,~J.~C., Stumpe,~M.~C., Van Cleve,~J.~E., \etal, 2012,
\Kepler\ Presearch Data Conditioning II.\ A Bayesian Approach to Systematic Error Correction,
\pasp, 124, 1000

\bibitem[Stumpe \etal(2012)]{map-pdc1}
Stumpe,~M.~C., Smith,~J.~C., Van Cleve,~J.~E., \etal, 2012,
\Kepler\ Presearch Data Conditioning I.\ Architecture and Algorithms for Error Correction in \Kepler\ Light Curves,
\pasp, 124, 985

\end{thebibliography}

\clearpage

\begin{figure}
\includegraphics[width=0.9\textwidth]{../whitepaper/kepler-20.png}%
\caption{Results of our data-driven pixel-level systematics model applied to
the quarter 9 observations of \Kepler-20 and compared to the PDC light curve.
\textsl{(top)}~The basic model (equation~\ref{eq:reg-model}) applied to the
pixel time series. This \figurename\ shows the results of coadding the
\emph{residuals} ($f_i (t_k)$ in equation~\ref{eq:reg-model}) using the
optimal aperture from the \Kepler\ pipeline.
\textsl{(middle)}~The same as the top panel using an autoregressive model
(with $\Delta = 12\,\mathrm{hours}$).
\textsl{(bottom)}~The results of doing simple aperture photometry and then
running PDC on the extracted light curve. \label{fig:reg-model}}
\end{figure}

\end{document}
